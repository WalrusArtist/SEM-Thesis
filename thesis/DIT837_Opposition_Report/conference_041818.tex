\documentclass[conference]{IEEEtran}
\IEEEoverridecommandlockouts
% The preceding line is only needed to identify funding in the first footnote. If that is unneeded, please comment it out.
\usepackage{cite}
\usepackage{amsmath,amssymb,amsfonts}
\usepackage{algorithmic}
\usepackage{graphicx}
\usepackage{textcomp}
\usepackage{xcolor}
\def\BibTeX{{\rm B\kern-.05em{\sc i\kern-.025em b}\kern-.08em
    T\kern-.1667em\lower.7ex\hbox{E}\kern-.125emX}}
\begin{document}

\title{Bachelor's Thesis Opposition Report}

\author{Your name(s):  Kardo Marof and Saif Sayed\\ \\
Thesis opposed (title): \textbf{Quality Assurance and Testing within Game Development:} \\
			       \quad\textit{Comparison between AAA and AA/Indie}
}

\maketitle


\section{Summary of the Bachelor Thesis}
The video game industry has seen remarkable growth in recent years, both in revenue and the number of consumers. Major digital PC video game distributor Steam recorded a peak of 33 million concurrent players in 2023, reflecting the increasing popularity of gaming. According to Clement, the industry's profits exceeded 400 billion USD in 2023. Despite the success of AAA studios like Dice, Creative Assembly, and Infinity Ward, there has been a significant shift in consumer perception due to the perceived decline in the quality of AAA game releases. This research aims to investigate the reasons behind this shift by comparing the QA and testing strategies of AAA and Indie/AA studios.

The thesis focuses on a comparative analysis of QA and testing practices between these two segments. By interviewing industry professionals from both AAA and Indie/AA studios, this study gathers firsthand insights and practical experiences, adding depth and relevance to the research. The findings could guide game development studios in refining their QA processes, potentially leading to higher quality game releases and better consumer satisfaction.

\subsection{Strengths of the Thesis}
The strengths of this thesis include its focused comparative analysis, providing detailed insights into the QA and testing practices of different game development scales. The empirical data collection through interviews with industry professionals offers practical relevance, while the potential practical implications could guide improvements in QA processes across the industry.

\subsection{Areas for Improvement}
However, there are areas for improvement. The study relies on a small sample size, which may not be representative of the entire industry. Expanding the number of interviewees and including a more diverse range of studios could enhance the reliability and generalizability of the findings. Additionally, the focus on QA and testing strategies overlooks other critical factors such as game complexity, development timelines, and market pressures. A more holistic approach could provide a comprehensive understanding of the issues. Incorporating multiple data sources and methods, such as quantitative surveys or case studies, could strengthen the validity of the conclusions and provide a more balanced perspective.


\section{Title and abstract}
The title of the Bachelor's Thesis reflects very well that the study demonstrated a comparison between AAA and AA/Indie studios, addressed from the perspective of Quality Assurance and Testing.

The abstract of the thesis was able to give the reader a summary of all the sections and the study. It included a brief overview of the background, the research gap, what research methodology was used to address the gap in knowledge, and finally the results.

However, it would be better to explicitly indicate that AAA studios represent larger game development studios, while AA/Indie studios represent smaller game development studios, especially for the benefit of readers unfamiliar with gaming industry terminology.


\section{Introduction}
The introduction covers various aspects of the video game industry's current state, focusing on the growth, consumer trends, and the perceived decline in the quality of AAA games compared to AA and Indie games, in a way that can be understood by readers from fields outside of Software Engineering. While the points made are intriguing, there are several issues and oversights that need to be addressed for a more balanced and comprehensive analysis.
\subsection{Overgeneralization of AAA Studios' Quality Issues}
The paper states that there is a widespread decline in the quality of AAA games, primarily citing stability and bug issues. While there have been notable high-profile failures, this is not representative of the entire AAA sector. Many AAA games continue to deliver high-quality, bug-free experiences that receive critical acclaim and strong sales. 

\subsection{Simplification of AA and Indie Game Success}
The introduction also claims that AA and Indie games are being received better due to higher quality. While smaller studios often produce innovative and polished games, they also face significant challenges, including limited budgets, marketing constraints, and the pressure of financial risk. The success of AA and Indie games is not universal, and many such games do not achieve the critical or commercial success needed to sustain their developers.\\

These claims were reportedly supported by a survey analysis. However, the source of these claims or the details of the survey were not stated explicitly in the thesis. As a result, it is difficult to determine whether these claims about the changing quality trends are accurate and well-substantiated.

Furthermore, the thesis introduced the research problem, which focused on the worsening quality of AAA game studios and the improved quality of AA/Indie game studios, along with the motivations behind these trends. It then attributed these differences to the testing and quality assurance practices of the respective game development studios, and the research questions were based on this premise. However, it was initially unclear whether the thesis would directly address the research problem concerning the worsening quality of AAA studios and the improving quality of AA/Indie studios. Instead, the thesis seemed to be taking a comparative approach, examining AAA and AA/Indie studios from the perspective of quality assurance and testing.
\section{Background and\slash or related work}
Overall, the Related Work section provided the reader with an overview of the recent rise of Indie and AA games, as well as background knowledge related to testing practices within the game development industry. All of these aspects were relevant to the research study being presented. However, the Related Work section seemed somewhat short and could potentially be expanded upon. For example, it could have included more discussion on the recent trend of worsening quality in AAA game studios, which was a key part of the research problem introduced in the thesis. The following sections will point out some critiques related to the sub-sections of the Related Work section:\\
\subsubsection{Recent Rise of Indie \& AA Games} This section of the Related Work addresses the concern regarding how Indie/AA studios compete with AAA studios. The authors state, there are factors such as Indie Studios being smaller, lower budget, and more independent than AAA studios. It fails to give speculations to why there has been a "Recent Rise of Indie \& AA Games".\\
\subsubsection{Testing within Game Development} This section is concise. It refers to Racheva's concepts of two core types of testing within Game Development, QA and Playtesting.  
The authors later refer to Cho's findings that non-indie studios rely more on QA, and Inde Studios relying on Playtesting. Where these differences in practice are speculated to be why Indie games are more bug-free.
One key concern here is although, given the presupposition that Indie games are more bug-free, what should compel them to follow guides and practices performed by non-indie game studios? The paper fails to address this argument.






\section{Methods (research methodology)}
The research methodology used for this study was appropriate for effectively answering the research questions and conducting a comparative analysis between AAA and AA/Indie game studios. The data collection and analysis processes were well-defined and illustrated, in this section, using relevant figures. The use of inter-coder reliability was a good approach to ensure the coded themes were consistent and reliable. However, the approach presents some critical issues that undermine the study's validity and comprehensiveness.

The Methodology section mentions that the companies were chosen at random, but it does not provide details to how this randomness was ensured . One other potential drawback was the relatively small number of participants used in this study. Moreover, not all of the interviewees had roles directly related to testing within their companies, which may have resulted in differing opinions about AAA and AA/Indie studios. Since this study focused on the perspective of quality assurance and testing, it would have been helpful to clarify why responses from interviewees in other roles were considered valid for the research. Additionally, there was no information provided on how the interview samples were selected and what criteria the participants had to fulfill.

Furthermore, the study primarily uses semi-structured interviews to collect data, which may not capture the full scope of QA and testing practices across different studios. While qualitative data provides depth, it needs to be complemented with quantitative data to ensure a robust analysis. The lack of quantitative measures limits the ability to compare practices objectively and may introduce subjectivity into the analysis.

\section{Results and figures}
The Results section included figures depicting the themes that were derived from the thematic analysis of the interview responses for both AAA and AA/Indie studios. By examining these figures, the reader can get an overview of the key interview topics, which were then described in more detail within the text using example quotes. However, the numbering of the themes in the figures did not consistently match the numbering used in the textual description of the results.

Overall, the results were presented in a thorough and well-structured manner, and the research questions were answered comprehensively. The findings successfully demonstrated the main similarities and differences between AAA and AA/Indie studios, as well as the potential for transferring knowledge between them.

That said, the research problem was not fully addressed by the results. While the study revealed important insights about the current state of quality assurance practices in these studios, it did not conclusively explain why the quality of AAA studios is reportedly worsening while the quality of AA/Indie studios is improving. The findings even presented some contradictory evidence, where AAA studios were shown to have skilled QA teams and mature processes, yet the quality was still seen as declining. This disconnect between the research questions and the research problem remained unresolved in the Results section.


\section{Discussion}
The Discussion section provides a well-reasoned personal interpretation of the study's results. It offers some explanations for why the quality of AAA studios is perceived to be worsening. However, the reasons for the improvement in the quality of AA/Indie studios are not clarified yet.

Furthermore, the majority of the Discussion section does not reference previous studies or relevant related works that could provide additional context and support for the interpretations and conclusions drawn from the research.

\section{Conclusions}
The Conclusion section provides a concise overview of how the research was conducted and summarizes the key findings for both AAA and AA/Indie studios. It also discusses the potential for mutual learning and knowledge transfer between these two types of game development studios.

Furthermore, the authors acknowledge that based on their findings, future research could investigate how to more accurately estimate deadlines for large-scope gaming projects, while also ensuring effective team communication, positive team dynamics, and employee passion.


\section{References}
The cited sources in the thesis mainly provide background information and context, but do not include the findings of closely related studies. The claims made about the worsening quality of AAA studios and the improving quality of AA/Indie studios are not supported by references to previous research/articles/studies. Similarly, the indication that the difference can be attributed to testing and quality assurance (QA) practices isn't based on existing literature.

The Discussion section discussed the findings based on personal interpretation and insights from the results. Additionally, the thesis does not explain how the cited references relate to the current study, or how the research builds upon or addresses gaps in the existing literature.


\section{Structure of the thesis report}
The structure of the whole Bachelor thesis was well-organized, containing relevant information under the headings and not deviating from what is supposed to be covered in each section. Even within each section, the text was organized into further subsections to help the reader get an overview of what is included in each section. In particular, the Results section was well presented and discussed thoroughly with an organized structure. However, one aspect that could be improved is the consistency in the numbering between the figures and the text for the themes identified.

\section*{Appendix: Detailed comments (if any)}
In the first line of the Abstract, the word "bugginess" does not seem appropriate. It is likely that the authors meant to convey that recent larger-budget video games have shown lower quality, which cannot be easily explained. The term "bugginess" could be interpreted in various ways here.

In the second line of the Introduction section, it would sound better to start the sentence by "This is also reflected by..." instead of "That is also reflected by...".

In the second line of the second paragraph in the Introduction section, the phrase "thanks to" appears somewhat informal for inclusion in a scientific research document.

In the Research Methodology section, there were many places where the text was written in future tense. In the same section, at the beginning, the phrase "we will" was used too much and consecutively.

It would be beneficial for the reader if the references were clickable, allowing them to navigate directly to the referenced literature, table, or figure. Try to run the bibliography file (bib) separately before compiling the main LaTeX (tex) file. Ensure that the bibliography style being used supports this functionality.




\vspace{12pt}

\end{document}
