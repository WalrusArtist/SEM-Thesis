\documentclass[conference]{IEEEtran}
\IEEEoverridecommandlockouts
% The preceding line is only needed to identify funding in the first footnote. If that is unneeded, please comment it out.
\usepackage{cite}
\usepackage{amsmath,amssymb,amsfonts}
\usepackage{algorithmic}
\usepackage{graphicx}
\usepackage{textcomp}
\usepackage{xcolor}
\def\BibTeX{{\rm B\kern-.05em{\sc i\kern-.025em b}\kern-.08em
    T\kern-.1667em\lower.7ex\hbox{E}\kern-.125emX}}
\begin{document}

\title{Bachelor's Thesis Opposition Report}

\author{Your name(s): \\ \\
Thesis opposed (title):
}

\maketitle


\section{Summary of bachelor thesis}
\textit{When you first receive the bachelor thesis, it is recommended that you read it through once and focus on the wider context of the research. After your first reading, write one or two paragraphs summarizing what the bachelor thesis is about and how it adds to current knowledge in the field. Mention the strengths of the bachelor thesis, but also any problems that make you believe could be improved. These summary paragraphs are the start of your review, and they will demonstrate that you have read the bachelor thesis carefully.}

\section{Title and abstract}
\textit{The title and abstract are items that will help other readers to decide if they will read the bachelor thesis further. Abstracts must be a clear, short summary of the full bachelor thesis. It is important that the abstract is interesting and hold the reader?s attention. Some questions to ask yourself about the title and abstract: a) does the title accurately say what the study was about? If not, can you suggest a different title? b) does the abstract summarize the bachelor thesis? c) could the abstract be understood by readers outside of software engineering?}


\section{Introduction}
\textit{Similar to the abstract, the introduction tells the reader what the bachelor thesis will be about. However, unlike the abstract, the Introduction gives the background for the research problem/aim/objective/questions. While reviewing the Introduction, some questions to ask yourself: a) does the Introduction explain the background well enough that people outside Software Engineering can understand it? b) does the Introduction accurately describe current knowledge related to the study? c) does the Introduction contain unnecessary information? Can it be made more concise? d) are the reasons for performing the study clear? e) are the aims of the study clearly defined and consistent with the rest of the bachelor thesis? You must discuss the problem that the bachelor thesis intends to address, including motivation for the study.}


\section{Background and\slash or related work}
\textit{Discuss consideration of related work and/or background in the bachelor thesis. Have the authors missed any (key) references that would be important for the study? Is the related work sufficient for the bachelor thesis and the study? Have the authors described the needed background? Are all papers described clearly related to the thesis? Make suggestions (if possible) for additional, relevant references if necessary.}


\section{Methods (research methodology)}
\textit{The bachelor thesis?s methods (research methods, data collection methods, data analysis methods etc.) are one of the most important parts used to judge the overall quality of the bachelor thesis. In addition, the Methods section should give the readers enough information so that they can repeat the study. As an opponent, you should look for potential sources of bias in the way the study was designed and carried out, and for places where more explanation is needed. The specific types of information in a Methods section will vary depending on the used research methodology. However, some general rules that you can follow are: a) it should be clear from the Methods section how all of the data in the Results section were obtained, b) the study system should be clearly described, e.g. number of study subjects, how, when and where the subjects were recruited, what criteria subjects had to meet to be included, which software was tested/evaluated, why this software, etc. c) the data collection method(s) should be clearly described, d) how the data was analyzed should be descried, e) an established research method must be used and justified, f) all materials and instruments should be identified, g) are the limitations/validity threats of the study noted (either here or in the Discussion section)? If not, what limitations/validity threats have you found?  You must discuss the appropriateness of the used Methods.}


\section{Results and figures}
\textit{Readers will usually first look at the bachelor thesis?s title, abstract and results. Therefore, the results section including any figures and tables are some of the most important parts of the bachelor thesis. As an opponent, you should carefully examine the figures and tables to check that they accurately describe the results. Some suggested questions that you can ask yourself: a) Table headings and figure legends should be detailed enough that readers can understand the data without reading the main text, b) look for places where data are unnecessary repeated in figures, tables, or in the main text, c) are the research questions answered? d) does the results section include results that are not part of the stated research question(s)? Discuss the presentation of the results, including all graphical presentations.}

\section{Discussion}
\textit{In the discussion section the author(s) should interpret the results and place them in context of previous findings (related work). If the author(s) have not made these points clear as they should be, note this in your opposition report. Other questions to ask include: a) does the discussion fit with the aims of the bachelor thesis stated in the Introduction? b) is there any general background that belongs in the Introduction/Methods/Results section rather than the Discussion? c) have the authors adequately compared their findings with the findings of other studies? d) are the limitations/validity threats of the study noted (either here or in the Methods section)? If not, what limitations/validity threats have you found?}

\section{Conclusions}
\textit{In the conclusion section, the author(s) should summarize the main findings of the bachelor thesis and explain future research (if future research is not placed in its own section). Note that no new (compared to what has been reported in the bachelor thesis until the conclusion section) results or information should be presented. Some questions to ask: a) do the authors mention how the study?s results might influence future research? b) are the authors? conclusions supported by their data? c) have the authors overstated the importance of their findings?}


\section{References}
\textit{Pay attention to how the authors use references. Some questions to ask yourself regarding the use of references: a) are there places in the bachelor thesis where the authors need to cite a reference but haven?t? b) do the authors cite relevant previous studies and explain how they relate to the current results? c) are the cited studies recent enough to represent current knowledge? d) do the authors cite work of a variety of researchers/universities? e) do the authors cite findings that contradict their own claims? It is important that the authors provide a well-balanced view of previously published work.}


\section{Structure of the thesis report}
\textit{Discuss the organization and structure of the bachelor thesis (aim, goal, methods, results, conclusions, etc...).}



\section*{Appendix: Detailed comments (if any)}
\textit{For example, the used language, typos, spelling mistakes, grammatical errors, poorly formatted references, etc. }

\vspace{12pt}

\end{document}
