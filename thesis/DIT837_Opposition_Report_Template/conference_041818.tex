\documentclass[conference]{IEEEtran}
\IEEEoverridecommandlockouts
% The preceding line is only needed to identify funding in the first footnote. If that is unneeded, please comment it out.
\usepackage{cite}
\usepackage{amsmath,amssymb,amsfonts}
\usepackage{algorithmic}
\usepackage{graphicx}
\usepackage{textcomp}
\usepackage{xcolor}
\def\BibTeX{{\rm B\kern-.05em{\sc i\kern-.025em b}\kern-.08em
    T\kern-.1667em\lower.7ex\hbox{E}\kern-.125emX}}
\begin{document}

\title{Bachelor's Thesis Opposition Report}

\author{Your name(s):  Kardo Marof and Saif Sayed\\ \\
Thesis opposed (title): \textbf{Quality Assurance and Testing within Game Development:} \\
			       \quad\textit{Comparison between AAA and AA/Indie}
}

\maketitle


\section{Summary of bachelor thesis}
\textit{When you first receive the bachelor thesis, it is recommended that you read it through once and focus on the wider context of the research. After your first reading, write one or two paragraphs summarizing what the bachelor thesis is about and how it adds to current knowledge in the field. Mention the strengths of the bachelor thesis, but also any problems that make you believe could be improved. These summary paragraphs are the start of your review, and they will demonstrate that you have read the bachelor thesis carefully.}

\section{Title and abstract}
The title of the Bachelor's Thesis reflects very well that the study demonstrated a comparison between AAA and AA/Indie studios, addressed from the perspective of Quality Assurance and Testing.

The abstract of the thesis was able to give the reader a summary of all the sections and the study. It included a brief overview of the background, the research gap, what research methodology was used to address the gap in knowledge, and finally the results.

However, it would be better to explicitly indicate that AAA studios represent larger game development studios, while AA/Indie studios represent smaller game development studios, especially for the benefit of readers unfamiliar with gaming industry terminology.


\section{Introduction}
The introduction covers various aspects of the video game industry's current state, focusing on the growth, consumer trends, and the perceived decline in the quality of AAA games compared to AA and Indie games, in a way that can be understood by readers from fields outside of Software Engineering. While the points made are intriguing, there are several issues and oversights that need to be addressed for a more balanced and comprehensive analysis.
\subsection{Overgeneralization of AAA Studios' Quality Issues}
The paper states that there is a widespread decline in the quality of AAA games, primarily citing stability and bug issues. While there have been notable high-profile failures, this is not representative of the entire AAA sector. Many AAA games continue to deliver high-quality, bug-free experiences that receive critical acclaim and strong sales. 

\subsection{Simplification of AA and Indie Game Success}
The introduction also claims that AA and Indie games are being received better due to higher quality. While smaller studios often produce innovative and polished games, they also face significant challenges, including limited budgets, marketing constraints, and the pressure of financial risk. The success of AA and Indie games is not universal, and many such games do not achieve the critical or commercial success needed to sustain their developers.\\

These claims were reportedly supported by a survey analysis. However, the source of these claims or the details of the survey were not stated explicitly in the thesis. As a result, it is difficult to determine whether these claims about the changing quality trends are accurate and well-substantiated.

Furthermore, the thesis introduced the research problem, which focused on the worsening quality of AAA game studios and the improved quality of AA/Indie game studios, along with the motivations behind these trends. It then attributed these differences to the testing and quality assurance practices of the respective game development studios, and the research questions were based on this premise. However, it was initially unclear whether the thesis would directly address the research problem concerning the worsening quality of AAA studios and the improving quality of AA/Indie studios. Instead, the thesis seemed to be taking a comparative approach, examining AAA and AA/Indie studios from the perspective of quality assurance and testing.
\section{Background and\slash or related work}
Overall, the Related Work section provided the reader with an overview of the recent rise of Indie and AA games, as well as background knowledge related to testing practices within the game development industry. All of these aspects were relevant to the research study being presented. However, the Related Work section seemed somewhat short and could potentially be expanded upon. For example, it could have included more discussion on the recent trend of worsening quality in AAA game studios, which was a key part of the research problem introduced in the thesis. The following sections will point out some critiques related to the sub-sections of the Related Work section:\\
\subsubsection{Recent Rise of Indie \& AA Games} This section of the Related Work addresses the concern regarding how Indie/AA studios compete with AAA studios. The authors state, there are factors such as Indie Studios being smaller, lower budget, and more independent than AAA studios. It fails to give speculations to why there has been a "Recent Rise of Indie \& AA Games".\\
\subsubsection{Testing within Game Development} This section is concise. It refers to Racheva's concepts of two core types of testing within Game Development, QA and Playtesting.  
The authors later refer to Cho's findings that non-indie studios rely more on QA, and Inde Studios relying on Playtesting. Where these differences in practice are speculated to be why Indie games are more bug-free.
One key concern here is although, given the presupposition that Indie games are more bug-free, what should compel them to follow guides and practices performed by non-indie game studios? The paper fails to address this argument.




\section{Methods (research methodology)}
The research methodology used for this study was appropriate for effectively answering the research questions and conducting a comparative analysis between AAA and AA/Indie game studios. The data collection and analysis processes were well-defined and illustrated, in this section, using relevant figures. The use of inter-coder reliability was a good approach to ensure the coded themes were consistent and reliable. However, the approach presents some critical issues that undermine the study's validity and comprehensiveness.

The Methodology section mentions that the companies were chosen at random, but it does not provide details to how this randomness was ensured . One other potential drawback was the relatively small number of participants used in this study. Moreover, not all of the interviewees had roles directly related to testing within their companies, which may have resulted in differing opinions about AAA and AA/Indie studios. Since this study focused on the perspective of quality assurance and testing, it would have been helpful to clarify why responses from interviewees in other roles were considered valid for the research. Additionally, there was no information provided on how the interview samples were selected and what criteria the participants had to fulfill.

Furthermore, the study primarily uses semi-structured interviews to collect data, which may not capture the full scope of QA and testing practices across different studios. While qualitative data provides depth, it needs to be complemented with quantitative data to ensure a robust analysis. The lack of quantitative measures limits the ability to compare practices objectively and may introduce subjectivity into the analysis.

\section{Results and figures}
The Results section included figures depicting the themes that were derived from the thematic analysis of the interview responses for both AAA and AA/Indie studios. By examining these figures, the reader can get an overview of the key interview topics, which were then described in more detail within the text using example quotes. However, the numbering of the themes in the figures did not consistently match the numbering used in the textual description of the results.

Overall, the results were presented in a thorough and well-structured manner, and the research questions were answered comprehensively. The findings successfully demonstrated the main similarities and differences between AAA and AA/Indie studios, as well as the potential for transferring knowledge between them.

That said, the research problem was not fully addressed by the results. While the study revealed important insights about the current state of quality assurance practices in these studios, it did not conclusively explain why the quality of AAA studios is reportedly worsening while the quality of AA/Indie studios is improving. The findings even presented some contradictory evidence, where AAA studios were shown to have skilled QA teams and mature processes, yet the quality was still seen as declining. This disconnect between the research questions and the research problem remained unresolved in the Results section.


\section{Discussion}
\textit{In the discussion section the author(s) should interpret the results and place them in context of previous findings (related work). If the author(s) have not made these points clear as they should be, note this in your opposition report. Other questions to ask include: a) does the discussion fit with the aims of the bachelor thesis stated in the Introduction? b) is there any general background that belongs in the Introduction/Methods/Results section rather than the Discussion? c) have the authors adequately compared their findings with the findings of other studies? d) are the limitations/validity threats of the study noted (either here or in the Methods section)? If not, what limitations/validity threats have you found?}

\section{Conclusions}
\textit{In the conclusion section, the author(s) should summarize the main findings of the bachelor thesis and explain future research (if future research is not placed in its own section). Note that no new (compared to what has been reported in the bachelor thesis until the conclusion section) results or information should be presented. Some questions to ask: a) do the authors mention how the study?s results might influence future research? b) are the authors? conclusions supported by their data? c) have the authors overstated the importance of their findings?}


\section{References}
\textit{Pay attention to how the authors use references. Some questions to ask yourself regarding the use of references: a) are there places in the bachelor thesis where the authors need to cite a reference but haven?t? b) do the authors cite relevant previous studies and explain how they relate to the current results? c) are the cited studies recent enough to represent current knowledge? d) do the authors cite work of a variety of researchers/universities? e) do the authors cite findings that contradict their own claims? It is important that the authors provide a well-balanced view of previously published work.}


\section{Structure of the thesis report}
\textit{Discuss the organization and structure of the bachelor thesis (aim, goal, methods, results, conclusions, etc...).}



\section*{Appendix: Detailed comments (if any)}
\textit{For example, the used language, typos, spelling mistakes, grammatical errors, poorly formatted references, etc. }

\vspace{12pt}

\end{document}
