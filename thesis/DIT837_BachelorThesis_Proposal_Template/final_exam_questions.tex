\documentclass[conference]{IEEEtran}
\IEEEoverridecommandlockouts
% The preceding line is only needed to identify funding in the first footnote. If that is unneeded, please comment it out.
\usepackage{cite}
\usepackage{amsmath,amssymb,amsfonts}
\usepackage{algorithmic}
\usepackage{graphicx}
\usepackage{textcomp}
\usepackage{xcolor}
\def\BibTeX{{\rm B\kern-.05em{\sc i\kern-.025em b}\kern-.08em
    T\kern-.1667em\lower.7ex\hbox{E}\kern-.125emX}}
\begin{document}

\title{DIT832 Exam Questions - 2024}

\author{Name of student 1: Kardo Marof\\
Name of student 2: Saif Sayed\\ \\
}

\maketitle


\section{Question 1}

\textbf{When designing a case study, there are four different types of case studies, depending on the purpose of the research and whether it is knowledge- or solution-focused. Imagine a case study that focuses on a team which may or may not apply a new type of Agile method, Scrumpfy. Briefly describe the four types of case studies. For each type, come up with one research question for the case study described.} \\

\textbf{Answer:} A case study can be either: \\

1) Exploratory: This type of case study aims to better understand the phenomenon being studied, look for new findings and generate ideas and hypothesis for new research. It facilitates a knowledge-focused research.\\

Example RQ for the suggested case study: How do the team perceive the effectiveness of the current Agile method being used?\\

2) Descriptive: This type of case study aims to provide a comprehensive description of a phenomenon or a situation. It facilitates a knowledge-focused research.\\

Example RQ for the suggested case study: What are the key characteristics and practices of the team with the current Agile methodology being used?\\

3) Explanatory: This type of case study  seeks an explanation of a phenomenon or a situation mostly by investigating casual relationships applicable for the context. It also facilitates a knowledge-focused research.\\

Example RQ for the suggested case study: What are the main issues with the current Agile method that indicates a need for a new Agile method?\\

4) Improving: This type of case study focuses on improving a certain aspect of a phenomenon being studied. It facilitates a solution-focused research.\\

Example RQ for the suggested case study: To what extent would the adoption of the Scrumpfy Agile method lead to improved team productivity and collaboration?

\section{Question 2}

\textbf{Design Science and Action Research have several similarities, but also some key differences. Describe the similarities and differences, focusing particularly on what makes these two methods different.}\\

\textbf{Answer:}

\section{Question 4}

\textbf{Imagine an experimental setup evaluating the effects of developer experience on program quality. We want to evaluate developers with low experience in programming ($<2$ years full time), medium experience (2 to 5 years), and high experience ($>$ 5 years). For such an experiment, what are the independent variables (factors), levels, dependent variables, objects, subjects, and possible parameters (control variables)?}\\

\textbf{Answer:}

\section{Question 5}

\textbf{Some MSR studies are experiments, while others claim to be experiments but do not follow experimental conventions in software engineering. What determines whether an MSR study is or is not an experiment?}\\

\textbf{Answer:}

\section{Question 6}

\textbf{What is the difference between a literature review, a Systematic Literature Review, and a Systematic Mapping Study?}\\

\textbf{Answer:}

\section{Question 7}

\textbf{As part of your planned thesis work, what sort of ethical issues may arise and why? What are you planning to do to mitigate ethical issues?}\\

\textbf{Answer:} 

\section{Acknowledgement}
The collaborative work of Saif Sayed and Kardo Marof resulted in the answers provided for all six questions. Both authors independently worked on their own answers for each question and then consolidated them into a final result.

\end{document}
