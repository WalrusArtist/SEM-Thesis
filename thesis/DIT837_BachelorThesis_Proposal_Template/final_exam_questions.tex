\documentclass[conference]{IEEEtran}
\IEEEoverridecommandlockouts
% The preceding line is only needed to identify funding in the first footnote. If that is unneeded, please comment it out.
\usepackage{cite}
\usepackage{amsmath,amssymb,amsfonts}
\usepackage{algorithmic}
\usepackage{graphicx}
\usepackage{textcomp}
\usepackage{xcolor}
\def\BibTeX{{\rm B\kern-.05em{\sc i\kern-.025em b}\kern-.08em
    T\kern-.1667em\lower.7ex\hbox{E}\kern-.125emX}}
\begin{document}

\title{DIT832 Exam Questions - 2024}

\author{Name of student 1: Kardo Marof\\
Name of student 2: Saif Sayed\\ \\
}

\maketitle


\section{Question 1}

\textbf{When designing a case study, there are four different types of case studies, depending on the purpose of the research and whether it is knowledge- or solution-focused. Imagine a case study that focuses on a team which may or may not apply a new type of Agile method, Scrumpfy. Briefly describe the four types of case studies. For each type, come up with one research question for the case study described.} \\

\textbf{Answer:} A case study can be either: \\

1) Exploratory: This type of case study aims to better understand the phenomenon being studied, look for new findings and generate ideas and hypothesis for new research. It facilitates a knowledge-focused research.\\

Example RQ for the suggested case study: How do the team perceive the effectiveness of the current Agile method being used?\\

2) Descriptive: This type of case study aims to provide a comprehensive description of a phenomenon or a situation. It facilitates a knowledge-focused research.\\

Example RQ for the suggested case study: What are the key characteristics and practices of the team with the current Agile methodology being used?\\

3) Explanatory: This type of case study  seeks an explanation of a phenomenon or a situation mostly by investigating casual relationships applicable for the context. It also facilitates a knowledge-focused research.\\

Example RQ for the suggested case study: What are the main issues with the current Agile method that indicates a need for a new Agile method?\\

4) Improving: This type of case study focuses on improving a certain aspect of a phenomenon being studied. It facilitates a solution-focused research.\\

Example RQ for the suggested case study: To what extent would the adoption of the Scrumpfy Agile method lead to improved team productivity and collaboration?

\section{Question 2}

\textbf{Design Science and Action Research have several similarities, but also some key differences. Describe the similarities and differences, focusing particularly on what makes these two methods different.}\\

\textbf{Answer:} First and foremost, both Design Science and Action Research are solution-focused approaches where the goal is to improve an aspect of a practical situation or a phenomenon. Secondly, they follow the build-measure-learn cycle iteratively to collect feedback and refine their approach based on it. This requires stakeholders to be involved for both of the research processes.\\

However, both of the methodologies have a different approach and outcomes to their final solution. Design Science helps researchers to create new solution to address specific problems in the form of designed artifacts, models and frameworks. In contrast, Action research focuses on understanding challenges in team practices to identify and implement changes in real-time through colloborative effort from teams, managers, etc. The outcome of Design Science is an innovative artifact or solution that can be used in practice while the outcome of Action Research is actionable knowledge that can be used to inform and guide improvements on a practice being studied.\\

Finally, Design Science allows researchers to develop artifacts that can be applied in different contexts while Action Research focuses on understanding and improving a particular practice within a specific context.

\section{Question 4}

\textbf{Imagine an experimental setup evaluating the effects of developer experience on program quality. We want to evaluate developers with low experience in programming ($<2$ years full time), medium experience (2 to 5 years), and high experience ($>$ 5 years). For such an experiment, what are the independent variables (factors), levels, dependent variables, objects, subjects, and possible parameters (control variables)?}\\

\textbf{Answer:} \\

\textbf{Independent variables (factors):} Developer Experience\\

\textbf{Levels:} \\
1) Low Experience: ($<2$ years full time)\\
2) Medium Experience: (2 to 5 years)\\
3) High Experience: ($>$ 5 years)\\

\textbf{Dependent variables:}  Program quality

\textbf{Objects:} Software Program

\textbf{Subjects:} Developers

\textbf{Control variables:} Task complexity, Programming language, Development Environment, Education level.

\section{Question 5}

\textbf{Some MSR studies are experiments, while others claim to be experiments but do not follow experimental conventions in software engineering. What determines whether an MSR study is or is not an experiment?}\\

\textbf{Answer:} To determine whether an MSR study is an experiment, one can observe if experimental characteristic are involved within the MSR study. For example, an experiment invloves manipulating variables to observe effects on dependent variables. This may include changing parameters related to software development practices or analysing impacts. Furthermore, experimental studies often include a controlled group and an experimental group. The controlled group will remain unchanged to keep as a base reference parallel to the experimental group. An experiment also applies randomisation techniques to participants and artifacts, where in traditonal MSR this characteristic may not be prevalent. The degree to which these experimental characteristics and convention are prevelant in an MSR study can determine whether or not the study is an experiment.



\section{Question 6}

\textbf{What is the difference between a literature review, a Systematic Literature Review, and a Systematic Mapping Study?}\\

\textbf{Answer:}
The differences between a literature review, a systematic literature review, and a systematic mapping study lie in their methodologies, scope, analysis and objectives. A literature review usually has a wide scope and may contain the researchers own judgement in selecting the literature, while a systematic literatire review adheres to rigorous strucured methods. Meanwhile, a systematic mapping study can also have systematic searching and classifying literature, but the aim and focus will more be on mapping the research landscape rather than combining individual studies.
In terms of analysis, literature review usually provides a summary and a synthesis to relevant literature, but a systematic literature review will often include quantitative analysis techniques in them. This differentiates their objectives, because a literature review often provides the current state of knowledge while a systemaic literature review will provide evidence-based synthesis to inform decision-making and identify areas for future research. On the other hand, a systematic mapping study aims to identify research trends, gaps and emerging interest to guide future research direction. 



\section{Question 7}

\textbf{As part of your planned thesis work, what sort of ethical issues may arise and why? What are you planning to do to mitigate ethical issues?}\\

\textbf{Answer:} 
In our research we will be conducting MSR, surveys and inteviews. In terms of ethical issues, our focus is mainly in part of the interviews.

In conducting interviews for research, ethical issues may arise concerning informed consent, privacy, voluntary participation, respect for the participants and data handling. To mitigate these concerns, we will obtain informed consent, ensure anonymity and confidentiality, establish boundaries, provide support, employ ethical interview techniques, seek ethics review, and continuously reflect on and monitor ethical considerations throughout the research process. By implementing these strategies, we can conduct the interviews ethically and responsibly, prioritizing the rights and well-being of the participants.

Additionally with MSR, mining data from public software repositories would require the consent of the contributors responsible for the specific repository. In order to mitigate it, we could ensure we are in compliance with the licensing agreement of the repository. Or we could explicitly ask for permission from the owners.

\section{Acknowledgement}
The collaborative work of Saif Sayed and Kardo Marof resulted in the answers provided for all six questions. Both authors independently worked on their own answers for each question and then consolidated them into a final result.

\end{document}
